%------------------------------------------------------------------------------
%    PACKAGES AND DOCUMENT CONFIGURATIONS
%------------------------------------------------------------------------------
\documentclass[10pt,a4paper,sans]{moderncv}

\moderncvstyle{classic}
\moderncvcolor{blue}

\usepackage[scale=0.79]{geometry}

\usepackage{color}
\definecolor{myblue}{rgb}{0.21875, 0.44921875, 0.69921875}
\definecolor{mygray}{rgb}{0.20, 0.35, 0.50}
\definecolor{linkcolor}{rgb}{0.0, 0.0, 0.93359375}

\usepackage{comment}

% Date diff
\usepackage{datenumber}
\newcounter{datetoday}
\newcounter{diffdays}
\newcounter{diffmonths}
\newcounter{diffyears}
\newcommand{\difftoday}[3]{%
      \setmydatenumber{datetoday}{\the\year}{\the\month}{\the\day}%
      \setmydatenumber{diffdays}{#1}{#2}{#3}%
      \addtocounter{diffdays}{-\thedatetoday}%
      \ifnum\value{diffdays}=0
        \def\diffbefore{in }%
        \def\diffafter{}%
      \else
        \def\diffbefore{}%
        \setcounter{diffdays}{-\value{diffdays}}%
      \fi
      \setcounter{diffyears}{\value{diffdays}/365}%
      \setcounter{diffdays}{\value{diffdays}-365*\value{diffyears}}%
      \setcounter{diffmonths}{\value{diffdays}/30}%
      \setcounter{diffdays}{\value{diffdays}-30*\value{diffmonths}}%
      %
      \diffbefore%
      \ifnum\value{diffyears}=0
      \else
        \ifnum\value{diffyears}>1
            \thediffyears{\space}years,
        \else
            \thediffyears{\space}year,
        \fi
      \fi
      \ifnum\value{diffmonths}=0
      \else
        \ifnum\value{diffmonths}>1
            \thediffmonths{\space}months%
        \else
            \thediffmonths{\space}month%
        \fi
      \fi
      \diffafter%
}%

\usepackage[T1]{fontenc}
\usepackage[utf8]{inputenc}

% smileys
\usepackage{MnSymbol,wasysym}

%------------------------------------------------------------------------------
%   CONTENT
%------------------------------------------------------------------------------
\firstname{Andrei}
\familyname{Pavel}

\title{curriculum vitae}
\address{}{Bucharest, Romania}
\mobile{+40 (0)748 220 135}
\email{andrei.pavel@cti.pub.ro}
\homepage{github.com/andrei-pavel}{https://github.com/andrei-pavel}
\photo[96pt][0.1pt]{pictures/portrait}
\begin{comment}
supporter of code correctness, performance, legibility\\
\end{comment}
\quote{true-stack on Linux with C++, D, Dart, Flutter, Go, Rust\\
% `by the way I use Arch Linux' --- geek meme \smiley{}\\
% `Don't settle.' --- Steve Jobs
}

\begin{document}
\makecvtitle{}


\section{technical skills}

\cvitem{advanced}{{} \newline{}
  {\color{gray}   \textbf{Android}: Java, Kotlin} \newline{}
  {\color{gray}   \textbf{C}: C11, user-level, Arduino-dialect, OpenMP, MPI, pthread} \newline{}
  {\color{myblue} \textbf{C++}: C++2a, casting, constness, dynamic loading, initialization order, inheritance, linkage, memory management, pointers, references, move semantics, preproessor, synchronization in concurrent programming, templates, STL, boost, CUDA, OpenGL, autotools, meson, lcov, cppcheck, clang-tools} \newline{}
  {\color{myblue} \textbf{Databases}: Cassandra, CouchDB, MSSQL, MariaDB, PostgreSQL, Oracle, Sybase, Sysrepo, sqlite} \newline{}
  {\color{myblue} \textbf{Debugging}: cgdb, gdb, gdbgui, lldb, pdb, rr} \newline{}
  {\color{myblue} \textbf{Go}: goroutines, reflection, BLE, ORMs, websockets, socket.io} \newline{}
  {\color{myblue} \textbf{Operating systems}: Linux, bash, zsh, sed, ack, ripgrep, systemd, sys-v-init, vim} \newline{}
  {\color{gray}   \textbf{Markup}: bpml, css, dtd, json, html, latex, markdown, xml, xsd, xsl, yang} \newline{}
  {\color{myblue} \textbf{Protocols}: DHCP, DNS, TCP, UDP ip, iw, netconf, netopeer, nftables, ss, tc, tcpdump, Wireshark} \newline{}
  {\color{myblue} \textbf{Version control}: git} \newline{}
  {\color{myblue} \textbf{Virtualization}: Docker, docker-compose, Helm, Kubernetes, KVM, libvirt, virsh, QEMU} \newline{}
}

\cvitem{intermediate}{{} \newline{}
  {\color{gray}   \textbf{C\#}: Windows Forms} \newline{}
  {\color{myblue} \textbf{D}: mixins, templates} \newline{}
  {\color{myblue} \textbf{Flutter}: asynchrony, Dart, generics, isolates, mixins, Provider} \newline{}
  {\color{myblue} \textbf{Rust}: affine types, linkage, macros, mutability, object lifetime, unsafety} \newline{}
}

\begin{comment}
\cvitem{beginner}{{} \newline{}
  {\color{gray} \textbf{Functional}: Haskell, Scheme} \newline{}
  {\color{gray} \textbf{Java}: JDBC} \newline{}
  {\color{gray} \textbf{Javascript}: Angular.js, node.js} \newline{}
  {\color{gray} \textbf{php}: idiorm} \newline{}
}
\end{comment}


\section{work experience ({\difftoday{2014}{11}{01}})}

\cventry{\
  \center{}
  May 2016 --- present\\({\difftoday{2016}{05}{09}})
  \\\includegraphics[width=\hintscolumnwidth]{pictures/qualitance}
}{\color{myblue} Software Engineer (formerly Junior C++ Developer)}{\textbf{\textsc{Qualitance}}}{Romania}{}{\
  \textbf{Kea: open-source Linux DHCP server, C++, enhancements}\\
  Open-source contributions to CESNET, ISC, Sysrepo ({\color{linkcolor}\url{https://github.com/andrei-pavel}}).
  \begin{itemize}
    \item \textbf{Cassandra} ODM with multiple data sources
    \item \textbf{DHCPv6} features and features that help migrate to DHCPv6 e.g.\ lighweight-4o6 DHCP options
    \item \textbf{Robot Framework} automated system-testing
    \item live, persistent server reconfiguration
    \item integration with Sysrepo data source, NETCONF protocol, YANG model
  \end{itemize}
  \textbf{Runlock: smart-lock system, development \& deployment \& unit tests}
  \begin{itemize}
    \item BLE beacon advertising \& GATT communication
    \item Electronic access control of doorlocks using Assa Abloy protocol
    \item REST
  \end{itemize}
  \textbf{Bullguard: antivirus backend for Mac OS X, C++, unit tests}
  \begin{itemize}
    \item scanning \& quarantining of all file types
    \item updating the virus database
  \end{itemize}
}

\begin{comment}
\cventry{\
  \center
  April 2016\\(1 month)
  \\\includegraphics[width=\hintscolumnwidth]{pictures/luxoft}
}{\color{myblue} Software Developer}{\textbf{\textsc{Luxoft}}}{Romania}{}{\
  Ended commitment while on trial.
}
\end{comment}

\cventry{\
  \center{}
  August 2014 --- October 2015 (1 year, \\2 months)
  \\\includegraphics[width=\hintscolumnwidth]{pictures/misys}
}
{\color{myblue} Associate Software Engineer}
{\textbf{\textsc{Misys Financial Software}}}{Romania}{}{\
  Development of treasury and capital markets software
  \begin{itemize}
    \item Implemented support for Negative Rates in Money Market Rollover (3 months).
    \item Maintenance and bug-solving (12 months).
    \item Worked with services distributed using CORBA across Windows, UNIX, Solaris querying Oracle, MSSQL, Sybase databases.
    \item Developed business objects (C++) and front-end functionality (C\#) for Summit.
          \newline{}
  \end{itemize}
}

\cventry{\
  \center{}
  June 2013 --- October 2013 (3 months)
  \\\includegraphics[width=\hintscolumnwidth]{pictures/intel}
}{\color{myblue} Software Engineering Intern}{\textbf{\textsc{Intel\textsuperscript{\textregistered}}}}{Romania}{}{\
  Application development and performance optimizations in the high performance computing
  \newline{}
  Accomplishments:
  \begin{itemize}
    \item Contributed to the development of parallel applications designed to benchmark HPC devices.
    \item Benchmarked the Intel\textsuperscript{\textregistered} XeonPhi\textsuperscript{TM} coprocessor during the early days of it's release.
    \item Used C, C++, OpenMP, native threading, low level intrinsics for more fine grained optimizations, Intel\textsuperscript{\textregistered} vTune\textsuperscript{TM} Amplifier for profiling and characterization studies.
          \newline{}
  \end{itemize}
}


\section{portfolio}

\cventry{2019-2020}{Radio România Actualităti Podcasts}{Podcast app written in Dart with Flutter}{}{}
{{\color{linkcolor} \url{https://play.google.com/store/apps/details?id=ro.radioromaniaactualitati.podcasts}
      \newline{}}}

\cventry{2018--2020}{bash-boilerplate}{bootstrap your bash project with this handy boilerplate}{}{}
{{\color{linkcolor} \url{https://github.com/andrei-pavel/bash-boilerplate}
      \newline{}}}

\cventry{2018--2019}{go-boilerplate}{bootstrap your go project with this full-fledged boilerplate}{}{}
{{\color{linkcolor} \url{https://github.com/andrei-pavel/go-boilerplate}
      \newline{}}}

\cventry{2014}{Rapunzel}{simulation and rendering of hair in real time using GPGPU techniques}{}{}
{{\color{linkcolor} \url{https://github.com/andrei-pavel/rapunzel}
      \newline{}}}


\section{books}

\cventry{currently reading}{\textbf{The way to Go}}{by Ivo Balbaert}{}{}
{{\color{linkcolor} \url{https://www.goodreads.com/book/show/13553772-the-way-to-go}}}

\cventry{latest read}{\textbf{Seven Databases in Seven Weeks: A Guide to Modern Databases and the NoSQL Movement}}{by Eric Redmond}{}{}
{{\color{linkcolor} \url{https://www.goodreads.com/book/show/25334471-optimizing-software-in-c}}}

\cventry{more on Goodreads}{{\color{linkcolor} \url{https://www.goodreads.com/user/show/63238106-andrei-pavel}}}{}{}{}
{}


\section{studies}

%\cventry{2014--2016}{\color{myblue} Masters of Computer Science}{Politehnica University of Bucharest}{Romania}{}{Advanced Software Services}
%\cvitem{Thesis}{\emph{\textbf{Acceleration of OCR algorithms using GPGPU techniques}} \newline{}
%Acceleration of clustering methods in OCR algorithms using non-conventional architecture that is inherently available in a holistic system performing image acquisition. Still in progress.}
%\cvitem{Supervisors}{Professor Florin Rădulescu}

\cventry{2010--2014}{\color{myblue} Bachelor of Computer Science}{Politehnica University of Bucharest}{Romania}{}{Automatic Control and Computer Science with major in Computer Systems Architecture}
\cvitem{Thesis}{\emph{\textbf{Simulation and rendering of hair in real time using GPGPU techniques}} \newline{}
  an algorithmic approach to the problem of real time simulation and rendering of hair in a highly-parallel multi-core environment}


\begin{comment}
\section{languages}

\cvitemwithcomment{Mothertongue}{Romanian}{}
\cvitemwithcomment{Advanced}{English}{conversationally fluent}
\cvitemwithcomment{Basic}{French}{basic words and phrases only}
\end{comment}

\end{document}
